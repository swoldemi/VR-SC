\documentclass[11pt]{article}
\usepackage[top=.6in, bottom=.6in]{geometry}
\usepackage{graphicx, float, tikz}
\usepackage{amsmath, amssymb, amsthm}
\def\doubleunderline#1{\underline{\underline{#1}}}
\usepackage{enumerate}
\usepackage{color, soul}
\usepackage[normalem]{ulem}

\begin{document}
\title{\Large CS 4331-002: Special Topics in CS - Virtual Reality \\ Spring 2018 Student's Choice Presentation: Report \\ \Large Simon Woldemichael \\ Tuesday, February 6, 2018 \\}
\author{\LARGE \textbf{NVIDIA Holodeck}}
\date{}

\maketitle
\vspace{-1cm}
\hspace{30pt}
\begin{center}
\textbf{Abstract}\\

NVIDIA Corporation's Project Holodeck is a high-end, photorealistic virtual reality environment designed for collaborative prototyping and design. The interface was developed to become an industrial standard. Although it is less than a year old and has higher than average hardware requirements, it is already being used by several commercial companies and firms within today's industry.
\end{center}

\paragraph{Overview} ~ \par 
Announced at last years GPU Technology Conference, NVIDIA Holodeck, also aliased as Project Holodeck, is NVIDIA Corporations \hl{plug} at speeding up and improving the creativity process. The platform, utilizing NVIDIA's own middle to high-end GPUs, was developed for designers, peers and stakeholders to collaborate in a more 3D, virtual, environment \cite{WEBSITE:2} with a high degree of freedom and flexibility. The environment involves the loading of fully developed and highly intricate Computer Aided Design models, made with AutoDesk Max or Maya. With the help of the Holodeck, these high-detail highly precise models give designers all of the information they need to go from prototype to final product.

\paragraph{Where is it used?} ~ \par 
	Currently, the VR environment is publicly used by 3 companies \cite{WEBSITE:2}; \textit{Koenigsegg Automotive AB}, \textit{NASA}, and \textit{Kohn Pedersen Fox}.
	
\paragraph{How does it work?} ~ \par 

\paragraph{Why is this a good use of virtual reality?} ~ \par 

\paragraph{User Experience} ~ \par  

\paragraph{Conclusion} ~ \par 

% Begin bibliography page
\newpage
\bibliography{sources}
\bibliographystyle{ieeetr}
\end{document}